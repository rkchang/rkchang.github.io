
\documentclass[10pt]{article}
\usepackage{enumitem}
\usepackage{float}
\usepackage{hyperref}
\usepackage{ifsym}
\usepackage{fontawesome5}
\usepackage[default,oldstyle,scale=0.95]{opensans}
\hypersetup{
    colorlinks=true,       % false: boxed links; true: colored links
    urlcolor=blue        % color of external links
}
\input{glyphtounicode}
%Lower amount of white space around headers
\usepackage[compact]{titlesec}
\titlespacing\section{0pt}{0pt plus 0pt minus 0pt}{0pt plus 0pt minus 0pt}
\titlespacing*{\subsection}{0pt}{0pt plus 0pt minus 0pt}{0pt plus 0pt minus 0pt}
% spacing: how to read {12pt plus 4pt minus 2pt}
%           12pt is what we would like the spacing to be
%           plus 4pt means that TeX can stretch it by at most 4pt
%           minus 2pt means that TeX can shrink it by at most 2pt
%       This is one example of the concept of, 'glue', in TeX

% Ensure resume is parsable by ATS
\input{glyphtounicode}
\pdfgentounicode=1

\usepackage{geometry}
\geometry{margin=1in}
\setcounter{secnumdepth}{0}
\setlength{\parindent}{0pt}
\begin{document}
\begin{center}
    {\huge\textbf{Raymond Chang}} \\
    \faIcon{envelope} \href{mailto:rkchang@uwaterloo.ca}{rkchang@uwaterloo.ca}  \space\space
    \faIcon{github} \href{https://github.com/rkchang}{github.com/rkchang} \space\space
    \faIcon{linkedin} \href{https://linkedin.com/in/rkhchang}{linkedin.com/in/rkhchang} \\
    \faIcon{mouse-pointer} \href{https://raymondchang.ca/}{raymondchang.ca} \\ \end{center}

{\Large\textbf{Skills}}\space \hrulefill
\begin{itemize}[noitemsep]
    \item Proficient: Python, C++, C, Java, SQL, TypeScript
    \item Technologies: Git, SQLite, GNU Make, Spring, CMake, PostgreSQL,
    \item Selected course work:  Design \& Analysis of Algorithms, Performance Engineering, Information Retrieval, Compiler Construction, Operating Systems, Databases, Artificial Intelligence, Language Design and Implementation
\end{itemize}

\smallskip

{\Large\textbf{Education}}\space \hrulefill

\textbf{University of Waterloo} \hfill September 2022 - Present

Masters of Mathematics in Computer Science (Thesis)

\hfill

\textbf{University of Ottawa} \hfill September 2016 - April 2021

Honours Bachelor of Science in Computer Science (CO-OP) \\ Deans Honours List \\ Honours project - Modelling and verifying distributed leader election algorithms with TLA+.
\bigskip

{\Large\textbf{Work Experience}}\space \hrulefill

\textbf{Blackberry QNX} \hfill June 2020 - August 2020 \\
CoreOS Software Development Student
\begin{itemize}[noitemsep]
    \item Developed a program to detect illegal instructions in the kernel binary by using binary analysis tools
    \item Wrote Unit tests for the QNX Neutrino RTOS and QNX Hypervisor
\end{itemize}

\textbf{Blackberry QNX}
\hfill September 2019 - December 2019 \\ CoreOS Software Development Student
\begin{itemize}[noitemsep]
    \item Worked on a program to test and track new commits to Review Board
    \item Setup a QNX Hypervisor System with an Ubuntu guest to demonstrate its abilities to other teams
    \item Wrote Unit tests for the QNX Neutrino RTOS and QNX Hypervisor
\end{itemize}

\textbf{Ford Motor Company}
\hfill January 2019 - April 2019 \\ Telematics Control Unit Software Developer
\begin{itemize}[noitemsep]
    \item Developed features and Unit tests for C and C++ multi-threaded Linux applications
    \item Fixed bugs and mutithreading related issues found by code analysis tools such as Thread Sanitizer and Clang Static Analyzer
\end{itemize}

\textbf{Canadian Border Services Agency}
\hfill May 2018 - August 2018 \\ Junior Programmer
\begin{itemize}[noitemsep]
    \item Worked in a team of three to create a web application for the management of software development using Java, JSP, Hibernate, jQuery, SQL and Spring MVC
    \item Developed functional UI Mock-ups for the web application using HTML, CSS and JavaScript
    \item Wrote JUnit Tests for a separate reporting application
    \item Acquired the Enhanced Reliability Status level of security clearance
\end{itemize}

\newpage

{\Large\textbf{Projects}}\space \hrulefill

\textbf{Juice - Java Compiler - Course project for Compiler Construction} \hfill January 2023 - April 2023
\begin{itemize}[noitemsep]
    \item Developed a Java compiler targeting x86 with two teammates
    \item Consisted of 28753 lines of TypeScript and passed the majority of course provided test cases
    \item Ported and upgraded an existing Intermediate Representation (IR) interpreter from Java to TypeScript to improve the debugging process
    \item Made major design decisions about the IR and closely consulted with group members
    \item Wrote hundreds of Jest unit tests with final code coverage of 80\% and total test count of 628 Jest unit tests
    \item Source code available upon request
\end{itemize}

\textbf{Lettuce - New Programming Language and LLVM based compiler} \hfill July 2023 - August 2023
\begin{itemize}[noitemsep]
    \item Worked with a partner to create a new programming language and corresponding compiler using the LLVM compiler infrastructure
    \item Developed compiler in C++ and added an interpreter mode
    \item Created a new IR using MLIR
    \item Source code: \url{https://github.com/rkchang/mlidk}
\end{itemize}

\textbf{Document Search Engine - Course project for Information Retrieval} \hfill January 2020 - April 2020
\begin{itemize}[noitemsep]
    \item Worked with a partner to create a search engine and UI for searching through a collection of course listings and news articles by using the Flask web framework
    \item Source code: \url{https://github.com/rkchang/search-engine}, Final report available upon request
\end{itemize}

\textbf{DNS Server} \hfill December 2021 - Now
\begin{itemize}[noitemsep]
    \item Currently writing a DNS server in C++ using the ASIO library
    \item Source code: \url{https://github.com/rkchang/rcdns}
\end{itemize}

\end{document}
